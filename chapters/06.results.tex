%!TEX root = ../report.tex
%!TeX encoding = UTF-8
%!TeX spellcheck = en_US

\chapter{Results \& Evaluation}\label{ch:results-evaluation}

This chapter should describe to what extent the goal of the project was achieved.

It should demonstrate that the system works as intended. It includes comprehensible summaries of the results of all critical tests that were carried out.

This is also the place to describe the reasoning behind the tests to evaluate the system, what tests were executed, what the results are showing, and why these tests were selected.

All the results should be critically evaluated in the light of the tests, considering its strengths and weaknesses. It should present ideas to improve the solution in future works. \textbf{Remember}: no project is perfect, and even a project that has failed to deliver what was intended can achieve a good pass mark, if it is clear that the team have learned from their mistakes and difficulties.

This chapter also gives to team an opportunity to present a critical appraisal of the project as a whole.

All these information should be distributed across the following sections.

\section{Methodology}\label{sec:methodology}

What criteria have you used to evaluate your proposed solution? What hypotheses do your experiment aims to test?


\section{Data set}\label{sec:data-set}

What data set have you used to evaluate your solution? What are the dependent and the independent variables?

\section{Results \& Data Analysis}\label{sec:results-data-analysis}

Describe the quantitative results of your experiments. A graphical representation is usually better than tables.

\section{Discussion}\label{sec:discussion}

Are your hypotheses supported? What conclusions do the results support the strengths and weakness of your proposed solution compared to other existing approaches? How can the results be explained in terms of the underlying properties of the algorithm and/or the data?